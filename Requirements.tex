\documentclass{article}
\usepackage{hyperref}
\usepackage{url}

\begin{document}

\title{Animusic Pipe Dream Engineering Project}

\author{Christian Boyd}

\maketitle

\tableofcontents

\begin{abstract}
    This document is (currently) a high-level outline of the requirements needed to 
    produce a physical re-creation of the Animusic Pipe Dream animation \cite{AnimusicPipeDream}.
\end{abstract}

\section{Projectile automation and control}
At a very basic level, re-creating the Animusic Pipe Dream animation requires the ability
to shoot spherical projectiles toward a variety of targets.  To this end, the initial
robotics effort will be toward implementing a satisfactory projectile system.

\subsection{Current requirements}

At a basic level, control over projectile motion requires control over the magnitude and
direction of the projectile force.  I've currently split this into the impulse and
orientation of our projectile system.  Of course, we need to propel {\it something}, and
so an additional requirement is to identify the actual projectiles.

\subsubsection{Propulsion impulse control}
A first-pass at a reasonable impulse requirement is the ability to launch the chosen
projectile 2 feet at an orientation of 80 degrees relative to the ground.

The desired system should demonstrate this capability at least 99\% of the time.

\subsubsection{Propulsion orientation control}
A first-pass at a reasonable orientation requirement is the ability to orient the initial
projectile direction.

This is a requirement to orient the initial projectile within 1 degree, separately in the
azimuthal and polar angles, of a prescribed spherical orientation.

\subsubsection{Procure suitable, spherical projectiles}
{\it Finish}...

\subsection{Nice-to-haves or future considerations}

\subsubsection{Rapid loading}

\subsubsection{Short firing cycle}

\bibliographystyle{unsrt}

\bibliography{mybib}

\end{document}